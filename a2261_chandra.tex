A2261 has two \emph{Chandra} exposures of 10 and 25 ksec (ObsIDs 550 and 5007, respectively), which do not conclusively reveal any accretion onto a SMBH.  The X-ray flux is those data is dominated by 0.5--2\,keV emission from hot cluster gas, showing a cooled core \citep{2005MNRAS.359.1481B}.  Above 2\
,keV, there are is some emission, possibly a point source consistent with knot 4, though it is far too faint to reliably distinguish it from cluster gas emission.  Deeper \emph{Chandra} observations can better reveal low-level accretion onto a SMBH and has the angular resolution to localize a point source such that it can be determined whether it is in one of the knots or at the core photometric center.  For example, if the black hole has a mass of $10^{10}\,\msun$, and it is emitting in the 2--7\,kEv band at an Eddington fraction of $10^{-6}$ as a power-law with photon index $\Gamma = 1.9$, a 100 ksec observation would detect 5 photons in the band, sufficient for significant detection above the background.  Such an observation would also be sensitive to a bow shock created by a recoiling SMBH and thus could provide strong positive evidence if it exists.


@ARTICLE{2005MNRAS.359.1481B,
   author = {{Bauer}, F.~E. and {Fabian}, A.~C. and {Sanders}, J.~S. and 
	{Allen}, S.~W. and {Johnstone}, R.~M.},
    title = "{The prevalence of cooling cores in clusters of galaxies at z\~{} 0.15-0.4}",
  journal = {\mnras},
   eprint = {astro-ph/0503232},
 keywords = {surveys, galaxies: clusters: general, cooling flows, X-rays: galaxies: clusters},
     year = 2005,
    month = jun,
   volume = 359,
    pages = {1481-1490},
      doi = {10.1111/j.1365-2966.2005.08999.x},
   adsurl = {http://adsabs.harvard.edu/abs/2005MNRAS.359.1481B},
  adsnote = {Provided by the SAO/NASA Astrophysics Data System}
}
